% ex1.tex - Extended Math Problems for AI-Resistant Exams
% Compile with lualatex or xelatex

\documentclass[12pt]{article}

\usepackage{amsmath} % For math environments
\usepackage{amssymb} % For mathematical symbols
\usepackage{graphicx} % For including images
\usepackage{fontspec} % Allows use of system fonts
\usepackage{xcolor}   % For colored text/watermarks
\usepackage{eso-pic}  % For background images/text
\usepackage{tikz}     % For drawing patterns/textures
\usepackage{layouts}  % For page dimensions
\usepackage{enumitem} % For customizing lists
\usepackage{float}    % For better positioning of figures

% --- Adversarial Modifications Placeholder ---
%%WATERMARK_AREA%%

\begin{document}

\section*{Advanced Mathematics Examination Problems}

\begin{enumerate}[label=\textbf{Problem \arabic*.}, leftmargin=*]

% ========== MULTIVARIABLE CALCULUS (Exam 2) ==========
\item Compute the partial derivatives $f_{x}$, $f_{y}$, $f_{xx}$, $f_{xy}$, $f_{yy}$.
\begin{enumerate}[label=(\alph*)]
\item $f(x,y)= 2x^3y^2 + 3y$
\vspace{3cm}
\item $f(x,y) = x e^{2y^2}$
\vspace{3cm}
\end{enumerate}

\item An overly organized developer landscapes land into artificial hillocks whose elevation is described by the equation $h(x,y) = \sin(x) \cos(y)$. 

The developer then makes a road through the hillocks along the curve defined by setting $x=\frac{\pi}{2}$.  Find the slope of this road (in the positive $y$-direction) at $x=\frac{\pi}{2}$, $y=-\frac{\pi}{2}$.
\vspace{5cm}

% ========== DISCRETE MATH (Exam 4) ==========
\item The following graph shows a map of a historic neighborhood. Is it possible to plan a walking tour through the neighborhood such that each street is traversed exactly once? If so, give a sequence of locations for the walking tour to visit (can the tour begin and end at the same location, or must these be different?). If not, explain why the walking tour is not possible.

\begin{figure}[H]
\centering
\includegraphics[width=0.5\textwidth]{CoF AI paper exam 4 (Discrete math)/graphmap.png}
\caption{Map of a historic neighborhood}
\end{figure}
\vspace{3cm}

\item Carry out the Criss-Cross algorithm to find a Hamilton circuit in the following graph:
\begin{figure}[H]
\centering
\includegraphics[width=0.5\textwidth]{CoF AI paper exam 4 (Discrete math)/CrissCrossExam.png}
\caption{Graph for Hamilton circuit problem}
\end{figure}
\vspace{5cm}

\newpage

% ========== COMPLEX ANALYSIS (Exam 3) ==========
\item Evaluate the contour integral $\int_C \frac{e^z}{(z-\log(-1-i))^{17}} \, dz$, where $C$ runs twice counterclockwise around the circle of radius 100 centered at $z=0$.
\vspace{8cm}

\item Find the circulation and flux of the 2-dimensional fluid flow $\vec{\bf w}(x,y) = \langle \frac{x+2}{(x+2)^2+(y+\pi)^2} , \frac{y+\pi}{(x+2)^2+(y+\pi)^2} \rangle$ around/through a circle of radius 1 centered at $z=0$, oriented counterclockwise.
\vspace{8cm}

% ========== MACHINE LEARNING (Exam 1) ==========
\item Why is gradient descent necessary for training logistic regression but not always needed for linear regression?
\vspace{6cm}

\item Suppose you have trained a logistic regression model for a binary classification problem and obtained the following confusion matrix:
\begin{center}
\begin{tabular}{c|c|c}
     & Predicted 0 & Predicted 1 \\
    \hline
    Actual 0 & 50 & 10 \\
    Actual 1 & 5 & 35 \\
\end{tabular}
\end{center}
Compute accuracy, precision, and recall for class 1. Recall that: 
\begin{itemize}
\item \textbf{Precision}: "the proportion of all the model's positive classifications that are actually positive." 
\item \textbf{Recall}: "the proportion of all actual positives that were classified correctly as positives"
\end{itemize}
\vspace{6cm}

\end{enumerate}

\end{document}
